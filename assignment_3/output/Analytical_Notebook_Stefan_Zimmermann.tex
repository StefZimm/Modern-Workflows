\PassOptionsToPackage{unicode=true}{hyperref} % options for packages loaded elsewhere
\PassOptionsToPackage{hyphens}{url}
%
\documentclass[]{article}
\usepackage{lmodern}
\usepackage{amssymb,amsmath}
\usepackage{ifxetex,ifluatex}
\usepackage{fixltx2e} % provides \textsubscript
\ifnum 0\ifxetex 1\fi\ifluatex 1\fi=0 % if pdftex
  \usepackage[T1]{fontenc}
  \usepackage[utf8]{inputenc}
  \usepackage{textcomp} % provides euro and other symbols
\else % if luatex or xelatex
  \usepackage{unicode-math}
  \defaultfontfeatures{Ligatures=TeX,Scale=MatchLowercase}
\fi
% use upquote if available, for straight quotes in verbatim environments
\IfFileExists{upquote.sty}{\usepackage{upquote}}{}
% use microtype if available
\IfFileExists{microtype.sty}{%
\usepackage[]{microtype}
\UseMicrotypeSet[protrusion]{basicmath} % disable protrusion for tt fonts
}{}
\IfFileExists{parskip.sty}{%
\usepackage{parskip}
}{% else
\setlength{\parindent}{0pt}
\setlength{\parskip}{6pt plus 2pt minus 1pt}
}
\usepackage{hyperref}
\hypersetup{
            pdftitle={Analytical Notebook},
            pdfauthor={Stefan Zimmermann},
            pdfborder={0 0 0},
            breaklinks=true}
\urlstyle{same}  % don't use monospace font for urls
\usepackage[margin=1in]{geometry}
\usepackage{color}
\usepackage{fancyvrb}
\newcommand{\VerbBar}{|}
\newcommand{\VERB}{\Verb[commandchars=\\\{\}]}
\DefineVerbatimEnvironment{Highlighting}{Verbatim}{commandchars=\\\{\}}
% Add ',fontsize=\small' for more characters per line
\usepackage{framed}
\definecolor{shadecolor}{RGB}{248,248,248}
\newenvironment{Shaded}{\begin{snugshade}}{\end{snugshade}}
\newcommand{\AlertTok}[1]{\textcolor[rgb]{0.94,0.16,0.16}{#1}}
\newcommand{\AnnotationTok}[1]{\textcolor[rgb]{0.56,0.35,0.01}{\textbf{\textit{#1}}}}
\newcommand{\AttributeTok}[1]{\textcolor[rgb]{0.77,0.63,0.00}{#1}}
\newcommand{\BaseNTok}[1]{\textcolor[rgb]{0.00,0.00,0.81}{#1}}
\newcommand{\BuiltInTok}[1]{#1}
\newcommand{\CharTok}[1]{\textcolor[rgb]{0.31,0.60,0.02}{#1}}
\newcommand{\CommentTok}[1]{\textcolor[rgb]{0.56,0.35,0.01}{\textit{#1}}}
\newcommand{\CommentVarTok}[1]{\textcolor[rgb]{0.56,0.35,0.01}{\textbf{\textit{#1}}}}
\newcommand{\ConstantTok}[1]{\textcolor[rgb]{0.00,0.00,0.00}{#1}}
\newcommand{\ControlFlowTok}[1]{\textcolor[rgb]{0.13,0.29,0.53}{\textbf{#1}}}
\newcommand{\DataTypeTok}[1]{\textcolor[rgb]{0.13,0.29,0.53}{#1}}
\newcommand{\DecValTok}[1]{\textcolor[rgb]{0.00,0.00,0.81}{#1}}
\newcommand{\DocumentationTok}[1]{\textcolor[rgb]{0.56,0.35,0.01}{\textbf{\textit{#1}}}}
\newcommand{\ErrorTok}[1]{\textcolor[rgb]{0.64,0.00,0.00}{\textbf{#1}}}
\newcommand{\ExtensionTok}[1]{#1}
\newcommand{\FloatTok}[1]{\textcolor[rgb]{0.00,0.00,0.81}{#1}}
\newcommand{\FunctionTok}[1]{\textcolor[rgb]{0.00,0.00,0.00}{#1}}
\newcommand{\ImportTok}[1]{#1}
\newcommand{\InformationTok}[1]{\textcolor[rgb]{0.56,0.35,0.01}{\textbf{\textit{#1}}}}
\newcommand{\KeywordTok}[1]{\textcolor[rgb]{0.13,0.29,0.53}{\textbf{#1}}}
\newcommand{\NormalTok}[1]{#1}
\newcommand{\OperatorTok}[1]{\textcolor[rgb]{0.81,0.36,0.00}{\textbf{#1}}}
\newcommand{\OtherTok}[1]{\textcolor[rgb]{0.56,0.35,0.01}{#1}}
\newcommand{\PreprocessorTok}[1]{\textcolor[rgb]{0.56,0.35,0.01}{\textit{#1}}}
\newcommand{\RegionMarkerTok}[1]{#1}
\newcommand{\SpecialCharTok}[1]{\textcolor[rgb]{0.00,0.00,0.00}{#1}}
\newcommand{\SpecialStringTok}[1]{\textcolor[rgb]{0.31,0.60,0.02}{#1}}
\newcommand{\StringTok}[1]{\textcolor[rgb]{0.31,0.60,0.02}{#1}}
\newcommand{\VariableTok}[1]{\textcolor[rgb]{0.00,0.00,0.00}{#1}}
\newcommand{\VerbatimStringTok}[1]{\textcolor[rgb]{0.31,0.60,0.02}{#1}}
\newcommand{\WarningTok}[1]{\textcolor[rgb]{0.56,0.35,0.01}{\textbf{\textit{#1}}}}
\usepackage{graphicx,grffile}
\makeatletter
\def\maxwidth{\ifdim\Gin@nat@width>\linewidth\linewidth\else\Gin@nat@width\fi}
\def\maxheight{\ifdim\Gin@nat@height>\textheight\textheight\else\Gin@nat@height\fi}
\makeatother
% Scale images if necessary, so that they will not overflow the page
% margins by default, and it is still possible to overwrite the defaults
% using explicit options in \includegraphics[width, height, ...]{}
\setkeys{Gin}{width=\maxwidth,height=\maxheight,keepaspectratio}
\setlength{\emergencystretch}{3em}  % prevent overfull lines
\providecommand{\tightlist}{%
  \setlength{\itemsep}{0pt}\setlength{\parskip}{0pt}}
\setcounter{secnumdepth}{0}
% Redefines (sub)paragraphs to behave more like sections
\ifx\paragraph\undefined\else
\let\oldparagraph\paragraph
\renewcommand{\paragraph}[1]{\oldparagraph{#1}\mbox{}}
\fi
\ifx\subparagraph\undefined\else
\let\oldsubparagraph\subparagraph
\renewcommand{\subparagraph}[1]{\oldsubparagraph{#1}\mbox{}}
\fi

% set default figure placement to htbp
\makeatletter
\def\fps@figure{htbp}
\makeatother


\title{Analytical Notebook}
\author{Stefan Zimmermann}
\date{11 7 2020}

\begin{document}
\maketitle

\hypertarget{introduction}{%
\subsection{Introduction}\label{introduction}}

This analytical notebook is a project for the IPSDS course Modern
Workflow in Data Science. In the project assignment 3 the number of
Corona cases and the infection rate of specific will be examined and
explained with the help of corona data from
\href{https://github.com/CSSEGISandData/COVID-19/tree/master/csse_covid_19_data}{github}.
The analyitical notebook should create two illustrations and one linear
model:

\begin{enumerate}
\def\labelenumi{\arabic{enumi}.}
\tightlist
\item
  Overall change of number of Corona Cases
\item
  Overall change of infection rate in percent
\item
  A ml\_linear\_regression explaining the log of number of Corona cases
\end{enumerate}

To write this report we have to set-up RStudio server on Amazon Web
Services (AWS) and use Spark there. Instructions for setting up a
Rstudio server with AWS can be found
\href{https://towardsdatascience.com/how-to-run-rstudio-on-aws-in-under-3-minutes-for-free-65f8d0b6ccda}{here}.

\hypertarget{install-and-load-r-packages}{%
\subsection{Install and load
R-Packages}\label{install-and-load-r-packages}}

This function installs and loads the required R-packages for the
analytical report. The following packages are required for the report:

\begin{itemize}
\tightlist
\item
  readr (rectangular data like csv, tsv, and fwf)
\item
  knitr (engine for dynamic report generation with R)
\item
  dplyr (tools for data manipulation)
\item
  tidyr (tools for data manipulation)
\item
  sparklyr (R interface for Apache Spark)
\item
  ggplot2 (system for declaratively creating graphics)
\end{itemize}

\begin{Shaded}
\begin{Highlighting}[]
\CommentTok{# install and load packages}
\NormalTok{loadpackage <-}\StringTok{ }\ControlFlowTok{function}\NormalTok{(x)\{}
  \ControlFlowTok{for}\NormalTok{( i }\ControlFlowTok{in}\NormalTok{ x )\{}
    \CommentTok{#  require returns TRUE invisibly if it was able to load package}
    \ControlFlowTok{if}\NormalTok{( }\OperatorTok{!}\StringTok{ }\KeywordTok{require}\NormalTok{( i , }\DataTypeTok{character.only =} \OtherTok{TRUE}\NormalTok{ ) )\{}
      \CommentTok{#  If package was not able to be loaded then re-install}
      \KeywordTok{install.packages}\NormalTok{( i , }\DataTypeTok{dependencies =} \OtherTok{TRUE}\NormalTok{ )}
\NormalTok{    \}}
    \CommentTok{#  Load package (after installing)}
    \KeywordTok{library}\NormalTok{( i , }\DataTypeTok{character.only =} \OtherTok{TRUE}\NormalTok{ )}
\NormalTok{  \}}
\NormalTok{\}}

\CommentTok{# load packages}
\KeywordTok{loadpackage}\NormalTok{( }\KeywordTok{c}\NormalTok{(}\StringTok{"readr"}\NormalTok{, }\StringTok{"knitr"}\NormalTok{, }\StringTok{"dplyr"}\NormalTok{, }\StringTok{"tidyr"}\NormalTok{, }\StringTok{"sparklyr"}\NormalTok{, }\StringTok{"ggplot2"}\NormalTok{))}
\end{Highlighting}
\end{Shaded}

\hypertarget{load-covid-datasets}{%
\subsection{\texorpdfstring{Load
\href{https://github.com/CSSEGISandData/COVID-19/tree/master/csse_covid_19_data}{Covid-Datasets}}{Load Covid-Datasets}}\label{load-covid-datasets}}

The datasets
\href{https://raw.githubusercontent.com/CSSEGISandData/COVID-19/master/csse_covid_19_data/UID_ISO_FIPS_LookUp_Table.csv}{UID\_ISO\_FIPS\_LookUp\_Table.csv}
and
\href{https://raw.githubusercontent.com/CSSEGISandData/COVID-19/master/csse_covid_19_data/csse_covid_19_time_series/time_series_covid19_confirmed_global.csv}{time\_series\_covid19\_confirmed\_global.csv}
are loaded and then harmonised (e.g.~The ID variables must be
standardized) to merge both datasets. Since we are only interested in
countries and not in regions we keep only a subset of the dataset
without countries regions combinations. To better handle the data we
reshape the data with the date variables to long format. Since data
variables are easy to handle in R we tidy the date variables before
setting up the spark connection. Afterwards the Spark connection is
established and the data sets are uploaded to Spark.

\begin{Shaded}
\begin{Highlighting}[]
\CommentTok{# use url to current dataset}
\NormalTok{url_data1 <-}\StringTok{ }
\StringTok{"https://raw.githubusercontent.com/CSSEGISandData/COVID-19/master/csse_covid_19_data/UID_ISO_FIPS_LookUp_Table.csv"}
\NormalTok{url_data2 <-}\StringTok{ }
\StringTok{"https://raw.githubusercontent.com/CSSEGISandData/COVID-19/master/csse_covid_19_data/csse_covid_19_time_series/time_series_covid19_confirmed_global.csv"}

\CommentTok{# load datasets}
\NormalTok{data1 <-}\StringTok{ }\KeywordTok{read_csv}\NormalTok{(url_data1)}
\NormalTok{data2 <-}\StringTok{ }\KeywordTok{read_csv}\NormalTok{(url_data2)}

\CommentTok{# harmonise ID-Variables in both datasets}
\KeywordTok{names}\NormalTok{(data1)[}\KeywordTok{names}\NormalTok{(data1) }\OperatorTok{==}\StringTok{ "Long_"}\NormalTok{] <-}\StringTok{ "Long"}
\KeywordTok{names}\NormalTok{(data2)[}\KeywordTok{names}\NormalTok{(data2) }\OperatorTok{==}\StringTok{ "Country/Region"}\NormalTok{] <-}\StringTok{ "Country_Region"}
\KeywordTok{names}\NormalTok{(data2)[}\KeywordTok{names}\NormalTok{(data2) }\OperatorTok{==}\StringTok{ "Province/State"}\NormalTok{] <-}\StringTok{ "Province_State"}

\CommentTok{# Since we are only interested in countries and not in regions}
\CommentTok{# we keep only a subset of the dataset without regions. }
\NormalTok{data1 <-}\KeywordTok{subset}\NormalTok{(data1, }\KeywordTok{is.na}\NormalTok{(data1}\OperatorTok{$}\NormalTok{Province_State))}
\NormalTok{data2 <-}\KeywordTok{subset}\NormalTok{(data2, }\KeywordTok{is.na}\NormalTok{(data2}\OperatorTok{$}\NormalTok{Province_State))}

\CommentTok{# reshape the dataset with the date variable in long format}
\NormalTok{data2 <-}\StringTok{ }
\StringTok{  }\KeywordTok{reshape}\NormalTok{(}
    \DataTypeTok{data =} \KeywordTok{as.data.frame}\NormalTok{(data2),}
    \DataTypeTok{varying =} \KeywordTok{list}\NormalTok{(}\KeywordTok{names}\NormalTok{(data2)[}\DecValTok{5}\OperatorTok{:}\KeywordTok{length}\NormalTok{(data2)]),}
    \DataTypeTok{timevar =} \StringTok{"day"}\NormalTok{,}
    \DataTypeTok{v.names =} \StringTok{"count"}\NormalTok{,}
    \DataTypeTok{idvar =} \KeywordTok{c}\NormalTok{(}\StringTok{"Country_Region"}\NormalTok{),}
    \DataTypeTok{direction =} \StringTok{"long"}\NormalTok{,}
    \DataTypeTok{times =} \KeywordTok{names}\NormalTok{(data2)[}\DecValTok{5}\OperatorTok{:}\KeywordTok{length}\NormalTok{(data2)]}
\NormalTok{  )  }

\NormalTok{data2}\OperatorTok{$}\NormalTok{date <-}\StringTok{ }\KeywordTok{as.Date}\NormalTok{(data2}\OperatorTok{$}\NormalTok{day, }\DataTypeTok{format =} \StringTok{"%m/%d/%y"}\NormalTok{)}
\NormalTok{data2}\OperatorTok{$}\NormalTok{datecount <-}\StringTok{ }\NormalTok{data2}\OperatorTok{$}\NormalTok{date}\OperatorTok{-}\KeywordTok{min}\NormalTok{(data2}\OperatorTok{$}\NormalTok{date)}
\NormalTok{days <-}\StringTok{ }\KeywordTok{unique}\NormalTok{(data2}\OperatorTok{$}\NormalTok{day)}
\NormalTok{weeks <-}\StringTok{ }\NormalTok{days[}\KeywordTok{seq}\NormalTok{(}\DecValTok{1}\NormalTok{, }\KeywordTok{length}\NormalTok{(days), }\DecValTok{7}\NormalTok{)]}

\KeywordTok{write_csv}\NormalTok{(data1, }\DataTypeTok{path =} \StringTok{'../input/data1.csv'}\NormalTok{)}
\KeywordTok{write_csv}\NormalTok{(data2, }\DataTypeTok{path =} \StringTok{'../input/data2.csv'}\NormalTok{)}

\CommentTok{# setting up spark}
\NormalTok{sc <-}\StringTok{ }\KeywordTok{spark_connect}\NormalTok{(}\DataTypeTok{master =} \StringTok{"local"}\NormalTok{,}
                    \DataTypeTok{version =} \StringTok{"2.4.3"}\NormalTok{)}

\NormalTok{data1 <-}\StringTok{ }\KeywordTok{sdf_copy_to}\NormalTok{(sc, data1, }\DataTypeTok{overwrite =}\NormalTok{ T)}
\NormalTok{data2 <-}\StringTok{ }\KeywordTok{sdf_copy_to}\NormalTok{(sc, data2, }\DataTypeTok{overwrite =}\NormalTok{ T)}
\KeywordTok{src_tbls}\NormalTok{(sc)}
\end{Highlighting}
\end{Shaded}

\hypertarget{data-cleaning}{%
\subsection{Data Cleaning}\label{data-cleaning}}

Here the two data sets are merged. We select the needed variables and
save the merged datasets. Now the data set is limited to the countries
(Germany, Japan, United Kingdom, US, Brazil, Mexico) which are to be
analyzed. Within this process new variables are created that
logarithmise the number of corona cases, divide the number of corona
cases by 1000 and generate the infection rate.

\begin{Shaded}
\begin{Highlighting}[]
\NormalTok{data_merge <-}\StringTok{ }\KeywordTok{merge}\NormalTok{(data2, data1, }\DataTypeTok{by =} \StringTok{"Country_Region"}\NormalTok{)}

\NormalTok{data_merge <-}\StringTok{ }\NormalTok{data_merge }\OperatorTok\StringTok{ }
\StringTok{  }\KeywordTok{select}\NormalTok{(Country_Region, Country_Region, day, date, count, datecount, Population)}

\KeywordTok{write_csv}\NormalTok{(data_merge, }\DataTypeTok{path =} \StringTok{'../input/data_merge.csv'}\NormalTok{)}
\NormalTok{data_merge <-}\StringTok{ }\KeywordTok{sdf_copy_to}\NormalTok{(sc, data_merge, }\DataTypeTok{overwrite =}\NormalTok{ T)}

\CommentTok{# Germany,Japan, United Kingdom, US, Brazil, Mexico }
\NormalTok{my_data <-}\StringTok{ }\NormalTok{data_merge }\OperatorTok
\StringTok{  }\KeywordTok{filter}\NormalTok{(Country_Region}\OperatorTok{==}\StringTok{"Germany"} \OperatorTok{|}\StringTok{ }\NormalTok{Country_Region}\OperatorTok{==}\StringTok{"Mexico"} \OperatorTok{|}
\StringTok{         }\NormalTok{Country_Region}\OperatorTok{==}\StringTok{"United Kingdom"} \OperatorTok{|}\StringTok{ }\NormalTok{Country_Region}\OperatorTok{==}\StringTok{"US"} \OperatorTok{|}
\StringTok{         }\NormalTok{Country_Region}\OperatorTok{==}\StringTok{"Brazil"} \OperatorTok{|}\StringTok{ }\NormalTok{Country_Region}\OperatorTok{==}\StringTok{"China"} \OperatorTok{|}\StringTok{ }
\StringTok{         }\NormalTok{Country_Region}\OperatorTok{==}\StringTok{"Japan"}\NormalTok{) }\OperatorTok\StringTok{ }
\StringTok{  }\KeywordTok{mutate}\NormalTok{(}\DataTypeTok{rate =}\NormalTok{ (count}\OperatorTok{/}\NormalTok{Population)}\OperatorTok{*}\DecValTok{100}\NormalTok{) }\OperatorTok\StringTok{ }
\StringTok{  }\KeywordTok{mutate}\NormalTok{(}\DataTypeTok{count2 =} \KeywordTok{round}\NormalTok{((count}\OperatorTok{/}\DecValTok{1000}\NormalTok{))) }\OperatorTok\StringTok{ }
\StringTok{  }\KeywordTok{mutate}\NormalTok{(}\DataTypeTok{logcount =} \KeywordTok{log}\NormalTok{(count)) }\OperatorTok\StringTok{ }
\StringTok{  }\KeywordTok{mutate}\NormalTok{(}\DataTypeTok{logcount =} \KeywordTok{ifelse}\NormalTok{(count }\OperatorTok{==}\StringTok{ }\DecValTok{0}\NormalTok{, }\DecValTok{0}\NormalTok{, logcount)) }\OperatorTok\StringTok{ }
\StringTok{  }\KeywordTok{collect}\NormalTok{()}

\NormalTok{my_data <-}\StringTok{ }\KeywordTok{sdf_copy_to}\NormalTok{(sc, my_data, }\DataTypeTok{name =} \StringTok{"my_data"}\NormalTok{, }\DataTypeTok{overwrite =} \OtherTok{TRUE}\NormalTok{)}
\end{Highlighting}
\end{Shaded}

\hypertarget{plots}{%
\subsection{Plots}\label{plots}}

The following code produces a line chart that describes the increase in
corona cases over a certain period of time. The countries Brazil, Japan,
Mexico, Germany, USA and Great Britain are shown. The X-axis shows the
observation period and the Y-axis shows the absolute corona cases of a
country over time. It can be seen that in all countries the corona
pandemic becomes visible with proven cases from mid-March onwards, but
Japan does not show much increase and in Mexico the increase of proven
cases starts later. While Japan, Great Britain and Germany seem to have
the pandemic under control, the figures for Brazil and the USA are still
rising very rapidly. The growth appears exponential. In Mexico the
increase starts later but seems to follow the trend of Brazil and USA.

\begin{Shaded}
\begin{Highlighting}[]
\KeywordTok{ggplot}\NormalTok{(}\DataTypeTok{data=}\NormalTok{my_data, }\KeywordTok{aes}\NormalTok{(}\DataTypeTok{x=}\KeywordTok{reorder}\NormalTok{(day, count2), }\DataTypeTok{y=}\NormalTok{count2, }
                         \DataTypeTok{group=}\NormalTok{Country_Region, }\DataTypeTok{colour=}\NormalTok{Country_Region)) }\OperatorTok{+}
\StringTok{  }\KeywordTok{geom_point}\NormalTok{()}\OperatorTok{+}
\StringTok{  }\KeywordTok{geom_line}\NormalTok{()}\OperatorTok{+}
\StringTok{  }\KeywordTok{scale_x_discrete}\NormalTok{(}\DataTypeTok{limit =}\NormalTok{ weeks)}\OperatorTok{+}
\StringTok{  }\KeywordTok{theme}\NormalTok{(}\DataTypeTok{axis.text.x =} \KeywordTok{element_text}\NormalTok{(}\DataTypeTok{angle =} \DecValTok{90}\NormalTok{, }\DataTypeTok{hjust =} \DecValTok{1}\NormalTok{),}
        \DataTypeTok{axis.title=}\KeywordTok{element_blank}\NormalTok{(),}
        \DataTypeTok{axis.ticks =} \KeywordTok{element_blank}\NormalTok{(),}
        \DataTypeTok{strip.text =} \KeywordTok{element_blank}\NormalTok{(),}
        \DataTypeTok{panel.grid.major =} \KeywordTok{element_blank}\NormalTok{(), }
        \DataTypeTok{panel.grid.minor =} \KeywordTok{element_blank}\NormalTok{(), }
        \DataTypeTok{panel.border =} \KeywordTok{element_blank}\NormalTok{(), }
        \DataTypeTok{panel.background =} \KeywordTok{element_blank}\NormalTok{())}\OperatorTok{+}
\StringTok{ }\KeywordTok{labs}\NormalTok{(}\DataTypeTok{title =}  \StringTok{"Overall change of number of Corona Cases in TSD"}\NormalTok{,}
       \DataTypeTok{caption =} \StringTok{"CSSEGISandData/COVID-19"}\NormalTok{)}
\end{Highlighting}
\end{Shaded}

\includegraphics{../output/Analytical_Notebook_Stefan_Zimmermann_files/figure-latex/unnamed-chunk-4-1.pdf}

\begin{Shaded}
\begin{Highlighting}[]
\KeywordTok{ggsave}\NormalTok{(}\StringTok{"../output/total_change.png"}\NormalTok{)}
\end{Highlighting}
\end{Shaded}

The following code produces a line chart that describes the infection
rate in corona cases over a certain period of time. The countries
Brazil, Japan, Mexico, Germany, USA and Great Britain are shown. The
X-axis shows the observation period and the Y-axis shows the infection
rate of a country over time in percent. The infection rate considers the
total population of the country. In contrast to the absolute figures,
this makes it possible to compare the trends better. As we can see,
Brazil, Mexico and the USA still have a rising infection rate. In the
other countries the strong increase could be slowed down. Japan and
Germany seem to have been particularly successful in dealing with the
pandemic.

\begin{Shaded}
\begin{Highlighting}[]
\KeywordTok{ggplot}\NormalTok{(}\DataTypeTok{data=}\NormalTok{my_data, }\KeywordTok{aes}\NormalTok{(}\DataTypeTok{x=}\KeywordTok{reorder}\NormalTok{(day, rate), }\DataTypeTok{y=}\NormalTok{rate, }
                         \DataTypeTok{group=}\NormalTok{Country_Region, }\DataTypeTok{colour=}\NormalTok{Country_Region)) }\OperatorTok{+}
\StringTok{  }\KeywordTok{geom_point}\NormalTok{()}\OperatorTok{+}
\StringTok{  }\KeywordTok{geom_line}\NormalTok{()}\OperatorTok{+}
\StringTok{  }\KeywordTok{scale_x_discrete}\NormalTok{(}\DataTypeTok{limit =}\NormalTok{ weeks)}\OperatorTok{+}
\StringTok{  }\KeywordTok{ylim}\NormalTok{(}\DecValTok{0}\NormalTok{,}\DecValTok{2}\NormalTok{)}\OperatorTok{+}
\StringTok{  }\KeywordTok{theme}\NormalTok{(}\DataTypeTok{axis.text.x =} \KeywordTok{element_text}\NormalTok{(}\DataTypeTok{angle =} \DecValTok{90}\NormalTok{, }\DataTypeTok{hjust =} \DecValTok{1}\NormalTok{),}
        \DataTypeTok{axis.title=}\KeywordTok{element_blank}\NormalTok{(),}
        \DataTypeTok{axis.ticks =} \KeywordTok{element_blank}\NormalTok{(),}
        \DataTypeTok{strip.text =} \KeywordTok{element_blank}\NormalTok{(),}
        \DataTypeTok{panel.grid.major =} \KeywordTok{element_blank}\NormalTok{(), }
        \DataTypeTok{panel.grid.minor =} \KeywordTok{element_blank}\NormalTok{(), }
        \DataTypeTok{panel.border =} \KeywordTok{element_blank}\NormalTok{(), }
        \DataTypeTok{panel.background =} \KeywordTok{element_blank}\NormalTok{())}\OperatorTok{+}
\StringTok{ }\KeywordTok{labs}\NormalTok{(}\DataTypeTok{title =}  \StringTok{"Overall change of infection rate in percent"}\NormalTok{,}
       \DataTypeTok{caption =} \StringTok{"CSSEGISandData/COVID-19"}\NormalTok{)}
\end{Highlighting}
\end{Shaded}

\includegraphics{../output/Analytical_Notebook_Stefan_Zimmermann_files/figure-latex/unnamed-chunk-5-1.pdf}

\begin{Shaded}
\begin{Highlighting}[]
\KeywordTok{ggsave}\NormalTok{(}\StringTok{"../output/rate_change.png"}\NormalTok{)}
\end{Highlighting}
\end{Shaded}

\hypertarget{regression}{%
\subsection{Regression}\label{regression}}

In this chapter we run an ml\_linear\_regression explaining the log of
number of cases using: country, population size and day since the start
of the pandemic. Therefore we split the sample into training and test
data set and run the regression on the training data set.

\begin{Shaded}
\begin{Highlighting}[]
\NormalTok{partitions <-}\StringTok{ }\NormalTok{my_data }\OperatorTok
\StringTok{  }\KeywordTok{sdf_random_split}\NormalTok{(}\DataTypeTok{training =} \FloatTok{0.7}\NormalTok{, }\DataTypeTok{test =} \FloatTok{0.3}\NormalTok{, }\DataTypeTok{seed =} \DecValTok{1111}\NormalTok{)}

\NormalTok{my_data_training <-}\StringTok{ }\NormalTok{partitions}\OperatorTok{$}\NormalTok{training}
\NormalTok{my_data_test <-}\StringTok{ }\NormalTok{partitions}\OperatorTok{$}\NormalTok{test}

\NormalTok{lm_model <-}\StringTok{ }\NormalTok{my_data_training }\OperatorTok
\StringTok{  }\KeywordTok{ml_linear_regression}\NormalTok{(logcount }\OperatorTok{~}\StringTok{ }\NormalTok{Country_Region }\OperatorTok{+}\StringTok{ }\NormalTok{Population }\OperatorTok{+}\StringTok{ }\NormalTok{datecount,}
                       \DataTypeTok{standardization=}\OtherTok{FALSE}\NormalTok{) }

\NormalTok{lm_model}
\end{Highlighting}
\end{Shaded}

\begin{verbatim}
## Formula: logcount ~ Country_Region + Population + datecount
## 
## Coefficients:
##                   (Intercept)         Country_Region_Brazil 
##                  3.334861e-01                  2.204834e-01 
##        Country_Region_Germany         Country_Region_Mexico 
##                  1.478153e+00                 -6.029792e-01 
##             Country_Region_US Country_Region_United Kingdom 
##                  1.747382e+00                  1.479916e+00 
##                    Population                     datecount 
##                  3.825215e-09                  7.999951e-02
\end{verbatim}

The model contains eight coefficents and the models formula is logcount
\textasciitilde{} Country\_Region + Population + datecount The first
coefficient of the output is the intercept. The intercept, in this model
models, is 0.33. We can say that 0.33 is the unconditional expected mean
of log of number of Corona cases. Therefore the exponentiated value is
39.58 percent. For the Country of Brazil, we can say that for a change
from not Brazil to Brazil, we expected to see about 24.67 percent of
increase of the average of number of Corona cases. For a metric variable
for example datecount, we can say that for a one-unit increase in
datecount, we expected to see about 8.33 percent of increase of the
average of number of Corona cases.

\end{document}
