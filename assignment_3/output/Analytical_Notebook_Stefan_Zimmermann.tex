\documentclass[]{article}
\usepackage{lmodern}
\usepackage{amssymb,amsmath}
\usepackage{ifxetex,ifluatex}
\usepackage{fixltx2e} % provides \textsubscript
\ifnum 0\ifxetex 1\fi\ifluatex 1\fi=0 % if pdftex
  \usepackage[T1]{fontenc}
  \usepackage[utf8]{inputenc}
\else % if luatex or xelatex
  \ifxetex
    \usepackage{mathspec}
  \else
    \usepackage{fontspec}
  \fi
  \defaultfontfeatures{Ligatures=TeX,Scale=MatchLowercase}
\fi
% use upquote if available, for straight quotes in verbatim environments
\IfFileExists{upquote.sty}{\usepackage{upquote}}{}
% use microtype if available
\IfFileExists{microtype.sty}{%
\usepackage{microtype}
\UseMicrotypeSet[protrusion]{basicmath} % disable protrusion for tt fonts
}{}
\usepackage[margin=1in]{geometry}
\usepackage{hyperref}
\hypersetup{unicode=true,
            pdftitle={Analytical Notebook},
            pdfauthor={Stefan Zimmermann},
            pdfborder={0 0 0},
            breaklinks=true}
\urlstyle{same}  % don't use monospace font for urls
\usepackage{color}
\usepackage{fancyvrb}
\newcommand{\VerbBar}{|}
\newcommand{\VERB}{\Verb[commandchars=\\\{\}]}
\DefineVerbatimEnvironment{Highlighting}{Verbatim}{commandchars=\\\{\}}
% Add ',fontsize=\small' for more characters per line
\usepackage{framed}
\definecolor{shadecolor}{RGB}{248,248,248}
\newenvironment{Shaded}{\begin{snugshade}}{\end{snugshade}}
\newcommand{\AlertTok}[1]{\textcolor[rgb]{0.94,0.16,0.16}{#1}}
\newcommand{\AnnotationTok}[1]{\textcolor[rgb]{0.56,0.35,0.01}{\textbf{\textit{#1}}}}
\newcommand{\AttributeTok}[1]{\textcolor[rgb]{0.77,0.63,0.00}{#1}}
\newcommand{\BaseNTok}[1]{\textcolor[rgb]{0.00,0.00,0.81}{#1}}
\newcommand{\BuiltInTok}[1]{#1}
\newcommand{\CharTok}[1]{\textcolor[rgb]{0.31,0.60,0.02}{#1}}
\newcommand{\CommentTok}[1]{\textcolor[rgb]{0.56,0.35,0.01}{\textit{#1}}}
\newcommand{\CommentVarTok}[1]{\textcolor[rgb]{0.56,0.35,0.01}{\textbf{\textit{#1}}}}
\newcommand{\ConstantTok}[1]{\textcolor[rgb]{0.00,0.00,0.00}{#1}}
\newcommand{\ControlFlowTok}[1]{\textcolor[rgb]{0.13,0.29,0.53}{\textbf{#1}}}
\newcommand{\DataTypeTok}[1]{\textcolor[rgb]{0.13,0.29,0.53}{#1}}
\newcommand{\DecValTok}[1]{\textcolor[rgb]{0.00,0.00,0.81}{#1}}
\newcommand{\DocumentationTok}[1]{\textcolor[rgb]{0.56,0.35,0.01}{\textbf{\textit{#1}}}}
\newcommand{\ErrorTok}[1]{\textcolor[rgb]{0.64,0.00,0.00}{\textbf{#1}}}
\newcommand{\ExtensionTok}[1]{#1}
\newcommand{\FloatTok}[1]{\textcolor[rgb]{0.00,0.00,0.81}{#1}}
\newcommand{\FunctionTok}[1]{\textcolor[rgb]{0.00,0.00,0.00}{#1}}
\newcommand{\ImportTok}[1]{#1}
\newcommand{\InformationTok}[1]{\textcolor[rgb]{0.56,0.35,0.01}{\textbf{\textit{#1}}}}
\newcommand{\KeywordTok}[1]{\textcolor[rgb]{0.13,0.29,0.53}{\textbf{#1}}}
\newcommand{\NormalTok}[1]{#1}
\newcommand{\OperatorTok}[1]{\textcolor[rgb]{0.81,0.36,0.00}{\textbf{#1}}}
\newcommand{\OtherTok}[1]{\textcolor[rgb]{0.56,0.35,0.01}{#1}}
\newcommand{\PreprocessorTok}[1]{\textcolor[rgb]{0.56,0.35,0.01}{\textit{#1}}}
\newcommand{\RegionMarkerTok}[1]{#1}
\newcommand{\SpecialCharTok}[1]{\textcolor[rgb]{0.00,0.00,0.00}{#1}}
\newcommand{\SpecialStringTok}[1]{\textcolor[rgb]{0.31,0.60,0.02}{#1}}
\newcommand{\StringTok}[1]{\textcolor[rgb]{0.31,0.60,0.02}{#1}}
\newcommand{\VariableTok}[1]{\textcolor[rgb]{0.00,0.00,0.00}{#1}}
\newcommand{\VerbatimStringTok}[1]{\textcolor[rgb]{0.31,0.60,0.02}{#1}}
\newcommand{\WarningTok}[1]{\textcolor[rgb]{0.56,0.35,0.01}{\textbf{\textit{#1}}}}
\usepackage{graphicx,grffile}
\makeatletter
\def\maxwidth{\ifdim\Gin@nat@width>\linewidth\linewidth\else\Gin@nat@width\fi}
\def\maxheight{\ifdim\Gin@nat@height>\textheight\textheight\else\Gin@nat@height\fi}
\makeatother
% Scale images if necessary, so that they will not overflow the page
% margins by default, and it is still possible to overwrite the defaults
% using explicit options in \includegraphics[width, height, ...]{}
\setkeys{Gin}{width=\maxwidth,height=\maxheight,keepaspectratio}
\IfFileExists{parskip.sty}{%
\usepackage{parskip}
}{% else
\setlength{\parindent}{0pt}
\setlength{\parskip}{6pt plus 2pt minus 1pt}
}
\setlength{\emergencystretch}{3em}  % prevent overfull lines
\providecommand{\tightlist}{%
  \setlength{\itemsep}{0pt}\setlength{\parskip}{0pt}}
\setcounter{secnumdepth}{0}
% Redefines (sub)paragraphs to behave more like sections
\ifx\paragraph\undefined\else
\let\oldparagraph\paragraph
\renewcommand{\paragraph}[1]{\oldparagraph{#1}\mbox{}}
\fi
\ifx\subparagraph\undefined\else
\let\oldsubparagraph\subparagraph
\renewcommand{\subparagraph}[1]{\oldsubparagraph{#1}\mbox{}}
\fi

%%% Use protect on footnotes to avoid problems with footnotes in titles
\let\rmarkdownfootnote\footnote%
\def\footnote{\protect\rmarkdownfootnote}

%%% Change title format to be more compact
\usepackage{titling}

% Create subtitle command for use in maketitle
\providecommand{\subtitle}[1]{
  \posttitle{
    \begin{center}\large#1\end{center}
    }
}

\setlength{\droptitle}{-2em}

  \title{Analytical Notebook}
    \pretitle{\vspace{\droptitle}\centering\huge}
  \posttitle{\par}
    \author{Stefan Zimmermann}
    \preauthor{\centering\large\emph}
  \postauthor{\par}
      \predate{\centering\large\emph}
  \postdate{\par}
    \date{11 7 2020}


\begin{document}
\maketitle

\hypertarget{install-and-load-r-packages}{%
\subsection{Install and load
R-Packages}\label{install-and-load-r-packages}}

\begin{Shaded}
\begin{Highlighting}[]
\CommentTok{# install and load packages}
\NormalTok{loadpackage <-}\StringTok{ }\ControlFlowTok{function}\NormalTok{(x)\{}
  \ControlFlowTok{for}\NormalTok{( i }\ControlFlowTok{in}\NormalTok{ x )\{}
    \CommentTok{#  require returns TRUE invisibly if it was able to load package}
    \ControlFlowTok{if}\NormalTok{( }\OperatorTok{!}\StringTok{ }\KeywordTok{require}\NormalTok{( i , }\DataTypeTok{character.only =} \OtherTok{TRUE}\NormalTok{ ) )\{}
      \CommentTok{#  If package was not able to be loaded then re-install}
      \KeywordTok{install.packages}\NormalTok{( i , }\DataTypeTok{dependencies =} \OtherTok{TRUE}\NormalTok{ )}
\NormalTok{    \}}
    \CommentTok{#  Load package (after installing)}
    \KeywordTok{library}\NormalTok{( i , }\DataTypeTok{character.only =} \OtherTok{TRUE}\NormalTok{ )}
\NormalTok{  \}}
\NormalTok{\}}

\CommentTok{# load packages}
\KeywordTok{loadpackage}\NormalTok{( }\KeywordTok{c}\NormalTok{(}\StringTok{"readr"}\NormalTok{, }\StringTok{"knitr"}\NormalTok{, }\StringTok{"dplyr"}\NormalTok{, }\StringTok{"tidyr"}\NormalTok{, }\StringTok{"sparklyr"}\NormalTok{, }\StringTok{"ggplot2"}\NormalTok{))}
\end{Highlighting}
\end{Shaded}

\hypertarget{load-covid-datasets}{%
\subsection{\texorpdfstring{Load
\href{https://github.com/CSSEGISandData/COVID-19/tree/master/csse_covid_19_data}{Covid-Datasets}}{Load Covid-Datasets}}\label{load-covid-datasets}}

The datasets
\href{https://raw.githubusercontent.com/CSSEGISandData/COVID-19/master/csse_covid_19_data/UID_ISO_FIPS_LookUp_Table.csv}{UID\_ISO\_FIPS\_LookUp\_Table.csv}
and
\href{https://raw.githubusercontent.com/CSSEGISandData/COVID-19/master/csse_covid_19_data/csse_covid_19_time_series/time_series_covid19_confirmed_global.csv}{time\_series\_covid19\_confirmed\_global.csv}
are loaded.

\begin{Shaded}
\begin{Highlighting}[]
\CommentTok{# use url to current dataset}
\NormalTok{url_data1 <-}\StringTok{ "https://raw.githubusercontent.com/CSSEGISandData/COVID-19/master/csse_covid_19_data/UID_ISO_FIPS_LookUp_Table.csv"}
\NormalTok{url_data2 <-}\StringTok{ "https://raw.githubusercontent.com/CSSEGISandData/COVID-19/master/csse_covid_19_data/csse_covid_19_time_series/time_series_covid19_confirmed_global.csv"}

\CommentTok{# load datasets}
\NormalTok{data1 <-}\StringTok{ }\KeywordTok{read_csv}\NormalTok{(url_data1)}
\NormalTok{data2 <-}\StringTok{ }\KeywordTok{read_csv}\NormalTok{(url_data2)}

\CommentTok{# harmonise ID-Variables in both datasets}
\KeywordTok{names}\NormalTok{(data1)[}\KeywordTok{names}\NormalTok{(data1) }\OperatorTok{==}\StringTok{ "Long_"}\NormalTok{] <-}\StringTok{ "Long"}
\KeywordTok{names}\NormalTok{(data2)[}\KeywordTok{names}\NormalTok{(data2) }\OperatorTok{==}\StringTok{ "Country/Region"}\NormalTok{] <-}\StringTok{ "Country_Region"}
\KeywordTok{names}\NormalTok{(data2)[}\KeywordTok{names}\NormalTok{(data2) }\OperatorTok{==}\StringTok{ "Province/State"}\NormalTok{] <-}\StringTok{ "Province_State"}

\CommentTok{# Since we are only interested in countries and not in regions}
\CommentTok{# we keep only a subset of the dataset without regions. }
\NormalTok{data1 <-}\KeywordTok{subset}\NormalTok{(data1, }\KeywordTok{is.na}\NormalTok{(data1}\OperatorTok{$}\NormalTok{Province_State))}
\NormalTok{data2 <-}\KeywordTok{subset}\NormalTok{(data2, }\KeywordTok{is.na}\NormalTok{(data2}\OperatorTok{$}\NormalTok{Province_State))}

\NormalTok{data2 <-}\StringTok{ }
\StringTok{  }\KeywordTok{reshape}\NormalTok{(}
    \DataTypeTok{data =} \KeywordTok{as.data.frame}\NormalTok{(data2),}
    \DataTypeTok{varying =} \KeywordTok{list}\NormalTok{(}\KeywordTok{names}\NormalTok{(data2)[}\DecValTok{5}\OperatorTok{:}\KeywordTok{length}\NormalTok{(data2)]),}
    \DataTypeTok{timevar =} \StringTok{"day"}\NormalTok{,}
    \DataTypeTok{v.names =} \StringTok{"count"}\NormalTok{,}
    \DataTypeTok{idvar =} \KeywordTok{c}\NormalTok{(}\StringTok{"Country_Region"}\NormalTok{),}
    \DataTypeTok{direction =} \StringTok{"long"}\NormalTok{,}
    \DataTypeTok{times =} \KeywordTok{names}\NormalTok{(data2)[}\DecValTok{5}\OperatorTok{:}\KeywordTok{length}\NormalTok{(data2)]}
\NormalTok{  )  }

\NormalTok{data2}\OperatorTok{$}\NormalTok{date <-}\StringTok{ }\KeywordTok{as.Date}\NormalTok{(data2}\OperatorTok{$}\NormalTok{day, }\DataTypeTok{format =} \StringTok{"%m/%d/%y"}\NormalTok{)}
\NormalTok{data2}\OperatorTok{$}\NormalTok{datecount <-}\StringTok{ }\NormalTok{data2}\OperatorTok{$}\NormalTok{date}\OperatorTok{-}\KeywordTok{min}\NormalTok{(data2}\OperatorTok{$}\NormalTok{date)}
\NormalTok{days <-}\StringTok{ }\KeywordTok{unique}\NormalTok{(data2}\OperatorTok{$}\NormalTok{day)}
\NormalTok{weeks <-}\StringTok{ }\NormalTok{days[}\KeywordTok{seq}\NormalTok{(}\DecValTok{1}\NormalTok{, }\KeywordTok{length}\NormalTok{(days), }\DecValTok{7}\NormalTok{)]}

\KeywordTok{write_csv}\NormalTok{(data1, }\DataTypeTok{path =} \StringTok{'../input/data1.csv'}\NormalTok{)}
\KeywordTok{write_csv}\NormalTok{(data2, }\DataTypeTok{path =} \StringTok{'../input/data2.csv'}\NormalTok{)}

\CommentTok{# setting up spark}
\NormalTok{sc <-}\StringTok{ }\KeywordTok{spark_connect}\NormalTok{(}\DataTypeTok{master =} \StringTok{"local"}\NormalTok{,}
                    \DataTypeTok{version =} \StringTok{"2.4.3"}\NormalTok{)}

\NormalTok{data1 <-}\StringTok{ }\KeywordTok{copy_to}\NormalTok{(sc, data1, }\DataTypeTok{overwrite =}\NormalTok{ T)}
\NormalTok{data2 <-}\StringTok{ }\KeywordTok{copy_to}\NormalTok{(sc, data2, }\DataTypeTok{overwrite =}\NormalTok{ T)}
\KeywordTok{src_tbls}\NormalTok{(sc)}
\end{Highlighting}
\end{Shaded}

\hypertarget{data-cleaning}{%
\subsection{Data Cleaning}\label{data-cleaning}}

Here the data sets are prepared for merging and reshaping. The ID
variables must be standardized. Since we are only interested in
countries and not in regions \# we keep only a subset of the dataset
without regions. Then we drop countries without information and we keep
all important variables. We reshape the dataset to long format

\begin{Shaded}
\begin{Highlighting}[]
\NormalTok{data_merge <-}\StringTok{ }\KeywordTok{merge}\NormalTok{(data2, data1, }\DataTypeTok{by =} \StringTok{"Country_Region"}\NormalTok{)}

\NormalTok{data_merge <-}\StringTok{ }\NormalTok{data_merge }\OperatorTok\StringTok{ }
\StringTok{  }\KeywordTok{select}\NormalTok{(Country_Region, Country_Region, day, date, count, datecount, Population)}

\KeywordTok{write_csv}\NormalTok{(data_merge, }\DataTypeTok{path =} \StringTok{'../input/data_merge.csv'}\NormalTok{)}
\NormalTok{data_merge <-}\StringTok{ }\KeywordTok{copy_to}\NormalTok{(sc, data_merge, }\DataTypeTok{overwrite =}\NormalTok{ T)}

\CommentTok{# Germany, China,Japan, United Kingdom, US, Brazil, Mexico }
\NormalTok{my_data <-}\StringTok{ }\NormalTok{data_merge }\OperatorTok
\StringTok{  }\KeywordTok{filter}\NormalTok{(Country_Region}\OperatorTok{==}\StringTok{"Germany"} \OperatorTok{|}\StringTok{ }\NormalTok{Country_Region}\OperatorTok{==}\StringTok{"Mexico"} \OperatorTok{|}
\StringTok{         }\NormalTok{Country_Region}\OperatorTok{==}\StringTok{"United Kingdom"} \OperatorTok{|}\StringTok{ }\NormalTok{Country_Region}\OperatorTok{==}\StringTok{"US"} \OperatorTok{|}
\StringTok{         }\NormalTok{Country_Region}\OperatorTok{==}\StringTok{"Brazil"} \OperatorTok{|}\StringTok{ }\NormalTok{Country_Region}\OperatorTok{==}\StringTok{"China"} \OperatorTok{|}\StringTok{ }
\StringTok{         }\NormalTok{Country_Region}\OperatorTok{==}\StringTok{"Japan"}\NormalTok{) }\OperatorTok\StringTok{ }
\StringTok{  }\KeywordTok{mutate}\NormalTok{(}\DataTypeTok{rate =}\NormalTok{ (count}\OperatorTok{/}\NormalTok{Population)}\OperatorTok{*}\DecValTok{100}\NormalTok{) }\OperatorTok\StringTok{ }
\StringTok{  }\KeywordTok{mutate}\NormalTok{(}\DataTypeTok{count2 =} \KeywordTok{round}\NormalTok{((count}\OperatorTok{/}\DecValTok{1000}\NormalTok{))) }\OperatorTok\StringTok{ }
\StringTok{  }\KeywordTok{collect}\NormalTok{() }\OperatorTok\StringTok{  }
\StringTok{  }\KeywordTok{print}\NormalTok{()}
\end{Highlighting}
\end{Shaded}

\begin{verbatim}
## # A tibble: 1,032 x 8
##    Country_Region day   date        count datecount Population   rate
##    <chr>          <chr> <date>      <dbl>     <dbl>      <dbl>  <dbl>
##  1 Brazil         6/21~ 2020-06-21 1.08e6       151  212559409 0.510 
##  2 Brazil         2/4/~ 2020-02-04 0.            13  212559409 0     
##  3 Brazil         2/8/~ 2020-02-08 0.            17  212559409 0     
##  4 Brazil         1/22~ 2020-01-22 0.             0  212559409 0     
##  5 Brazil         6/19~ 2020-06-19 1.03e6       149  212559409 0.486 
##  6 Brazil         2/6/~ 2020-02-06 0.            15  212559409 0     
##  7 Brazil         5/9/~ 2020-05-09 1.56e5       108  212559409 0.0734
##  8 Brazil         5/13~ 2020-05-13 1.90e5       112  212559409 0.0895
##  9 Brazil         5/4/~ 2020-05-04 1.09e5       103  212559409 0.0511
## 10 Brazil         5/29~ 2020-05-29 4.65e5       128  212559409 0.219 
## # ... with 1,022 more rows, and 1 more variable: count2 <dbl>
\end{verbatim}

\hypertarget{plots}{%
\subsection{Plots}\label{plots}}

You can also embed plots, for example:

\begin{Shaded}
\begin{Highlighting}[]
\KeywordTok{ggplot}\NormalTok{(}\DataTypeTok{data=}\NormalTok{my_data, }\KeywordTok{aes}\NormalTok{(}\DataTypeTok{x=}\KeywordTok{reorder}\NormalTok{(day, count2), }\DataTypeTok{y=}\NormalTok{count2, }\DataTypeTok{group=}\NormalTok{Country_Region, }\DataTypeTok{colour=}\NormalTok{Country_Region)) }\OperatorTok{+}
\StringTok{  }\KeywordTok{geom_point}\NormalTok{()}\OperatorTok{+}
\StringTok{  }\KeywordTok{geom_line}\NormalTok{()}\OperatorTok{+}
\StringTok{  }\KeywordTok{scale_x_discrete}\NormalTok{(}\DataTypeTok{limit =}\NormalTok{ weeks)}\OperatorTok{+}
\StringTok{  }\KeywordTok{theme}\NormalTok{(}\DataTypeTok{axis.text.x =} \KeywordTok{element_text}\NormalTok{(}\DataTypeTok{angle =} \DecValTok{90}\NormalTok{, }\DataTypeTok{hjust =} \DecValTok{1}\NormalTok{),}
        \DataTypeTok{axis.title=}\KeywordTok{element_blank}\NormalTok{(),}
        \DataTypeTok{axis.ticks =} \KeywordTok{element_blank}\NormalTok{(),}
        \DataTypeTok{strip.text =} \KeywordTok{element_blank}\NormalTok{(),}
        \DataTypeTok{panel.grid.major =} \KeywordTok{element_blank}\NormalTok{(), }
        \DataTypeTok{panel.grid.minor =} \KeywordTok{element_blank}\NormalTok{(), }
        \DataTypeTok{panel.border =} \KeywordTok{element_blank}\NormalTok{(), }
        \DataTypeTok{panel.background =} \KeywordTok{element_blank}\NormalTok{())}\OperatorTok{+}
\StringTok{ }\KeywordTok{labs}\NormalTok{(}\DataTypeTok{title =}  \StringTok{"Overall change of number of Corona Cases in TSD"}\NormalTok{,}
       \DataTypeTok{caption =} \StringTok{"CSSEGISandData/COVID-19"}\NormalTok{)}
\end{Highlighting}
\end{Shaded}

\begin{verbatim}
## Warning: Removed 882 rows containing missing values (geom_point).
\end{verbatim}

\begin{verbatim}
## Warning: Removed 882 rows containing missing values (geom_path).
\end{verbatim}

\includegraphics{../output/Analytical_Notebook_Stefan_Zimmermann_files/figure-latex/unnamed-chunk-61-1.pdf}

\begin{Shaded}
\begin{Highlighting}[]
\KeywordTok{ggsave}\NormalTok{(}\StringTok{"../output/total_change.png"}\NormalTok{)}
\end{Highlighting}
\end{Shaded}

\begin{verbatim}
## Saving 6.5 x 4.5 in image
\end{verbatim}

\begin{verbatim}
## Warning: Removed 882 rows containing missing values (geom_point).

## Warning: Removed 882 rows containing missing values (geom_path).
\end{verbatim}

\begin{Shaded}
\begin{Highlighting}[]
\KeywordTok{ggplot}\NormalTok{(}\DataTypeTok{data=}\NormalTok{my_data, }\KeywordTok{aes}\NormalTok{(}\DataTypeTok{x=}\KeywordTok{reorder}\NormalTok{(day, rate), }\DataTypeTok{y=}\NormalTok{rate, }\DataTypeTok{group=}\NormalTok{Country_Region, }\DataTypeTok{colour=}\NormalTok{Country_Region)) }\OperatorTok{+}
\StringTok{  }\KeywordTok{geom_point}\NormalTok{()}\OperatorTok{+}
\StringTok{  }\KeywordTok{geom_line}\NormalTok{()}\OperatorTok{+}
\StringTok{  }\KeywordTok{scale_x_discrete}\NormalTok{(}\DataTypeTok{limit =}\NormalTok{ weeks)}\OperatorTok{+}
\StringTok{  }\KeywordTok{ylim}\NormalTok{(}\DecValTok{0}\NormalTok{,}\DecValTok{2}\NormalTok{)}\OperatorTok{+}
\StringTok{  }\KeywordTok{theme}\NormalTok{(}\DataTypeTok{axis.text.x =} \KeywordTok{element_text}\NormalTok{(}\DataTypeTok{angle =} \DecValTok{90}\NormalTok{, }\DataTypeTok{hjust =} \DecValTok{1}\NormalTok{),}
        \DataTypeTok{axis.title=}\KeywordTok{element_blank}\NormalTok{(),}
        \DataTypeTok{axis.ticks =} \KeywordTok{element_blank}\NormalTok{(),}
        \DataTypeTok{strip.text =} \KeywordTok{element_blank}\NormalTok{(),}
        \DataTypeTok{panel.grid.major =} \KeywordTok{element_blank}\NormalTok{(), }
        \DataTypeTok{panel.grid.minor =} \KeywordTok{element_blank}\NormalTok{(), }
        \DataTypeTok{panel.border =} \KeywordTok{element_blank}\NormalTok{(), }
        \DataTypeTok{panel.background =} \KeywordTok{element_blank}\NormalTok{())}\OperatorTok{+}
\StringTok{ }\KeywordTok{labs}\NormalTok{(}\DataTypeTok{title =}  \StringTok{"Overall change of infection rate in percent"}\NormalTok{,}
       \DataTypeTok{caption =} \StringTok{"CSSEGISandData/COVID-19"}\NormalTok{)}
\end{Highlighting}
\end{Shaded}

\begin{verbatim}
## Warning: Removed 882 rows containing missing values (geom_point).

## Warning: Removed 882 rows containing missing values (geom_path).
\end{verbatim}

\includegraphics{../output/Analytical_Notebook_Stefan_Zimmermann_files/figure-latex/unnamed-chunk-61-2.pdf}

\begin{Shaded}
\begin{Highlighting}[]
\KeywordTok{ggsave}\NormalTok{(}\StringTok{"../output/rate_change.png"}\NormalTok{)}
\end{Highlighting}
\end{Shaded}

\begin{verbatim}
## Saving 6.5 x 4.5 in image
\end{verbatim}

\begin{verbatim}
## Warning: Removed 882 rows containing missing values (geom_point).

## Warning: Removed 882 rows containing missing values (geom_path).
\end{verbatim}

\hypertarget{regression}{%
\subsection{Regression}\label{regression}}


\end{document}
